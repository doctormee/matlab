\documentclass[11pt, oneside, draft]{article}
\usepackage[utf8]{inputenc}
\usepackage{a4wide}
\usepackage[russian]{babel}
\usepackage{graphicx}
\usepackage{epstopdf}
\usepackage{amsmath}
\usepackage{amsfonts}
\usepackage{amssymb}
\usepackage{amsthm}

%\newtheoremstyle{mytheorem}% name of the style to be used
%  {}% measure of space to leave above the theorem. E.g.: 3pt
%  {}% measure of space to leave below the theorem. E.g.: 3pt
%  {}% name of font to use in the body of the theorem
%  {}% measure of space to indent
%  {}% name of head font
%  {:}% punctuation between head and body
%  { }% space after theorem head; " " = normal interword space
%  {}% Manually specify head
%\theoremstyle{mytheorem}
\numberwithin{equation}{section}
\newtheorem{definition}{Определение}[section]
\newtheorem{theorem}{Теорема}[section]
\newtheorem{property}{Свойство}[section]
\newtheorem{corollary}{Следствие}[theorem]
\newtheorem{lemma}[theorem]{Лемма}
\newtheorem*{statement}{Утверждение}
\renewenvironment{proof}{\noindent\textit{Доказательство: }} {\qed}

%commands
\newcommand \bitem[1][]{\item \textbf{#1}}
\newcommand \four[1][\lambda]{\mathfrak{F}(#1)}
\newcommand \fft[1][\lambda]{F(#1)}
\newcommand \rarrow{\rightarrow}
\newcommand \intinf[1][{\,dt}]{ \int\limits_{-\infty}^{+\infty}{{#1}}}
\renewcommand \qed{$\blacksquare$}

\DeclareMathOperator{\sgn}{sgn}

\begin{document}
%Title
\thispagestyle{empty}

\begin{center}
\ \vspace{-3cm}

\includegraphics[width=0.5\textwidth]{msu}\\
{\scshape Московский государственный университет имени М.~В.~Ломоносова}\\
Факультет вычислительной математики и кибернетики\\
Кафедра системного анализа

\vfill

{\LARGE Отчёт по практикуму}

\vspace{1cm}

{\Huge\bfseries "<Быстрое преобразование Фурье">}
\end{center}

\vspace{1cm}

\begin{flushright}
  \large
  \textit{Студент 315 группы}\\
  В.\,А.~Сливинский

  \vspace{5mm}

  \textit{Руководители практикума}\\
 	к.ф.-м.н., доцент И.\,В.~Рублёв \\
    	к.ф.-м.н., доцент П.\,А.~Точилин
\end{flushright}

\vfill

\begin{center}
Москва, 2017
\end{center}
\pagebreak
%Contents
\tableofcontents

\pagebreak
%Task
\section{Постановка~задачи}
\subsection{Общая~формулировка~задачи}
Дана система функций (всюду далее, если не сказано противное, предполагается, что \(f(t) : \mathbb{R} \rightarrow \mathbb{R} \) и функция суммируема и обладает достаточной гладкостью) 
\begin{equation}\label{functions}
\left\{
\begin{aligned}  
	f_1(t) &= e^{-2|t|} \cos(t) \\
	f_2(t) &= \dfrac{e^{-|t|} - 1}{t} \\
	f_3(t) &= \dfrac{\arctg{t^2}}{1 + t^4} \\
	f_4(t) &= t^3e^{-t^4}
\end{aligned}
\right.
\end{equation}
Для каждой функции из системы \eqref{functions} требуется:

\begin{enumerate}
	\item Получить аппроксимацию преобразования Фурье  \( F(\lambda)\) для каждой функции \(f(t)\) из заданного набора при помощи быстрого преобразования Фурье ({\bfseries БПФ / FFT}), выбирая различные шаги 
дискретизации исходной функции и различные окна, ограничивающие область определения \(f(t)\)
	\item Построить графики \(F(\lambda)\)
	\item Для функций \(f_1(t)\) и \(f_2(t)\) из заданного набора вычислить аналитически преобразование Фурье 
	\begin{equation}
	\label{fourier_transform} 
	\boxed{\four = \intinf[{f(t) e^{-i\lambda t}\, dt}]}
	\end{equation}
	и сравнить графики  \( \four \) с графиками  \( F(\lambda)\), полученного из аппроксимации через {\bfseries БПФ}\\
\end{enumerate}
%Formal task
\subsection{Формальная~постановка~задачи}
\begin{enumerate}
	\item Реализовать на языке MATLAB функцию \\\texttt{plotFT(hFigure,~fHandle,~fFTHandle,~step,~inpLimVec,~outLimVec)} \\со следующими параметрами:
	\begin{itemize}
		\bitem[hFigure] ~--- указатель на фигуру, в которой требуется отобразить графики
		\bitem[fHandle] ~--- указатель на функцию (\texttt{Function Handle}), которую требуется преобразовывать (\(f(t)\))
		\bitem[fFTHandle] ~--- указатель на функцию (\texttt{Function Handle}), моделирующую аналитическое преобразование Фурье \eqref{fourier_transform} функции \(f(t)\) 
		(может быть пустым вектором, в таком случае график аналитического преобразования строить не требуется)
		\bitem[step] ~--- положительное число, задающее шаг дискретизации \(\Delta t\)
		\bitem[inpLimVector] ~--- вектор-строка, задающая окно \([a, b]\) для функции \(f(t)\), первый элемент вектора содержит \(a\), второй \(b\), причём~\(a < b\), но не обязательно \(a = -b\)
		\bitem[outLimVector] ~--- вектор-строка, задающая окно \([c, d]\) \textit{для вывода} графика преобразования Фурье (пределы~осей~абсцисс). 
		В случае, если передаётся пустой вектор, следует брать установленные в фигуре пределы или определять свои разумным образом
	\end{itemize}
	Данная функция строит графики вещественной и мнимой частей численной аппроксимации преобразования Фурье (\ref{fourier_transform}) функции \(f(t)\), заданной в \textbf{\texttt{fHandle}}
	 (и, при необходимости, соответствующие графики~аналитического~преобразования~Фурье~\(\four\)) \\
	 Кроме того, данная функция, должна возвращать структуру, содержащую следующие параметры: 
	\begin{itemize}
		\bitem[nPoints] ~--- число вычисляемых узлов сеточной функции, рассчитываемое по формуле: 
		\[ nPoints = \Bigl\lfloor {\dfrac{(b - a)}{step}}\Bigr\rfloor \]
		\bitem[step] ~--- поправленное значение шага дискретизации \(\Delta t\), рассчитываемое по формуле: \[step= \dfrac{(b - a)}{nPoints}\]
		\bitem[inpLimVec] ~---  окно \([a, b]\) для функции \(f(t)\)
		\bitem[outLimVec] ~--- окно для вывода графика преобразования Фурье \(\fft\)
	\end{itemize}
	\item Построить, используя написанную функцию \texttt{plotFT}, для каждой из функций системы \eqref{functions} графики \(\fft\) для разных значений входных параметров \\*
	(окон \textbf{inpLimVec, outLimVec} и частоты дискретизации \textbf{step}. \\В частности, для некоторых функций подобрать параметры так, чтобы проиллюстрировать
	эффекты \textit{наложения спектра, появления~ряби и их устранения} (в случае ряби~--- в точках непрерывности \(\fft\))
	\item Для функций \(f_1(t)\) и \(f_2(t)\) из системы \eqref{functions} вычислить аналитически их преобразования Фурье \(\four\) и построить их графики вместе с графиками численной
	аппроксимации \(\fft\)
\end{enumerate}

%fourier
\section{Вычисление~аналитических~преобразований~Фурье}

	\subsection{Некоторые~необходимые~обозначения~и~соотношения}
		Напомним, что преобразование Фурье \( \four \) функции \(f(t)\) задаётся формулой \eqref{fourier_transform}:
			\begin{align*} 
				\four = \intinf[ {f(t) e^{-i\lambda t}\, dt}]
			\end{align*}
		Впредь, будем для краткости писать: 
		\[
			\boxed{ f(t)\rarrow\four}
		\]
		
		
		{
		Напомним также следующие свойства преобразования Фурье:
		
		\begin{property}\label{property:linear}
		\mdseriesПусть  \begin{gather*} f(t) = \alpha \cdot f_1(t) + \beta \cdot f_2(t) \text{ , и }
		\left\{
            	 \begin{aligned}
            		f_1(t) &\rarrow \mathfrak{F_1}(\lambda) \\
            		f_2(t) &\rarrow \mathfrak{F_2}(\lambda)
            	\end{aligned}
		\right.
		\end{gather*} \\
		Тогда: \[
		 f(t) \rarrow  \alpha \cdot \mathfrak{F_1}(\lambda) + \beta \cdot  \mathfrak{F_2}(\lambda)  
		 \]
		\end{property}
		\begin{property}\label{property:product}
		\mdseriesПусть  \begin{gather*} f(t) = f_1(t) \cdot f_2(t) \text{ , и }
		\left\{
            	 \begin{aligned}
            		f_1(t) &\rarrow \mathfrak{F_1}(\lambda) \\
            		f_2(t) &\rarrow \mathfrak{F_2}(\lambda)
            	\end{aligned}
		\right.
		\end{gather*}\\
		Тогда: \[
		2\pi f_1(t) \cdot f_2(t) \rarrow  (\mathfrak{F_1 * F_2)}(\lambda) \text{ , где }(\mathfrak{F_1 * F_2})(\lambda) = \intinf[{\bigl[\mathfrak{F_1}(\lambda - s) \cdot \mathfrak{F_2}(s)\bigr] \, ds}]
		 \]
		\end{property}
		}
		
		
		{
		Отметим некоторые тривиальные преобразования Фурье:
		\begin{flalign}
		\label{fourier:delta}\delta(\lambda) &\rarrow 1 \\
		\label{fourier:1} 1 &\rarrow 2\pi\delta(\lambda) \\
		\label{fourier:exp} e^{iat} &\rarrow 2\pi\delta(\lambda - a) \\
		\label{fourier:cos} \cos(t) = \frac{e^{it} + e^{-it}}{2} & \rarrow \pi(\delta(\lambda - 1) + \delta(\lambda + 1)) \\
		\label{fourier:1/t} \dfrac{1}{t} &\rarrow -i \pi \sgn(t)
		\end{flalign}
		Где \( \delta(t) =
            				 \begin{cases}
		 			+\infty,& t = 0 \\
					0,& t \not= 0
            				\end{cases} 
					\)~--- дельта-функция Дирака,
		а соотношение \eqref{fourier:cos} вытекает из свойства~\ref{property:linear}, с учётом \eqref{fourier:exp}. 
		}
		
		%delta
		{
		Установим также важное отношения для свёртки дельта-функции с произвольной функцией \(\varphi(t)\):
		\begin{align}
		\label{delta:conv}\boxed{\left(\delta * \varphi \right) (s) = \intinf[{\delta(s - \tau) \cdot \varphi(\tau)\, d\tau} = \varphi(s)]}
		\end{align}
		}
		
		\noindent Докажем следующее соотношение:
		\begin{lemma}
		\begin{equation}\label{fourier:exp_abs}
		e^{-A|t|} \rarrow \dfrac{2A}{A^2 + \lambda^2}
		\end{equation}
		\end{lemma}
		\begin{proof}
		\[
		\begin{split} 
			\intinf[{e^{-A|t|} \cdot e^{-i\lambda t} \, dt}] &= \int\limits_{-\infty}^0{e^{(A - i\lambda) t} \, dt} +  \int\limits_0^{+\infty}{e^{-(A + i\lambda) t} \, dt} = \\
			&= \left[ e^{(A - i\lambda)t} \cdot \frac{1}{A - i\lambda} \right]_{t = -\infty}^{0} 
			- \left[ e^{-(A + i\lambda)t} \cdot \frac{1}{A + i\lambda} \right]_{t = 0}^{+\infty} = \\ 
			&=\frac{1}{A - i\lambda} + \frac{1}{A + i\lambda} = \frac{2A}{A^2 + \lambda^2}
		\end{split}
		\]
		\end{proof}
%Fourier 1
\subsection{Вычисление~аналитического~преобразования~Фурье\\функции~\(f_1(t) = e^{-2|t|} \cos(t)\)}

	Преобразование Фурье \( \mathfrak{F_1} (\lambda)\) функции \(f_1(t) = e^{-2|t|} \cos(t) \) задаётся формулой:
	\[
		\mathfrak{F_1} (\lambda) = \intinf[{e^{-2|t|} \cos(t) e^{-i\lambda t}\, dt}]
	\]
	
	\begin{statement}
	\begin{equation}\label{fourier_transform:f1}
	\boxed{
		\mathfrak{F_1}(\lambda) =  \dfrac{4(\lambda^2 + 5)}{\lambda^4 + 6\lambda^2 + 25}
	}
	\end{equation}
	\end{statement}
	
	\begin{proof}
	Заметим, что \(f_1(t) \) представима в виде: 
	\begin{equation}\label{factor:f1}
	f_1(t) = g_1(t) \cdot g_2(t) \text{, где }g_1(t) = e^{-2|t|},\,g_2(t) = \cos(t)
	\end{equation}
	Пользуясь этим соотношением, выражениями для преобразований Фурье \(g_1(t)\) \eqref{fourier:exp_abs} и \(g_2(t)\) \eqref{fourier:cos}, установленным свойством \ref{property:product}
	 и соотношением \eqref{delta:conv} для свёртки с дельта-функцией, получим: 
	 \[
	 \begin{split}
		 \mathfrak{F_1} (\lambda) &= \dfrac{1}{2\pi} \intinf[{\dfrac{4}{4 + \tau^2} \cdot \pi(\delta(\lambda - \tau- 1) + \delta(\lambda + 1 - \tau))\,d\tau}] = \\
		 &=\dfrac{2}{4 + (\lambda - 1)^2} + \dfrac{2}{4 + (\lambda + 1)^2} 
		= \dfrac{4(\lambda^2 + 5)}{\lambda^4 + 6\lambda^2 + 25}
	\end{split}
	\]
	\end{proof}
	
%Fourier 2
\subsection{Вычисление~аналитического~преобразования~Фурье\\функции~\(f_2(t) = \frac{e^{-|t|} - 1}{t} \)}
Преобразование Фурье \( \mathfrak{F_2} (\lambda)\) функции \(f_2(t) = \frac{e^{-|t|} - 1}{t} \) задаётся формулой:
	\[
		\mathfrak{F_2} (\lambda) = \intinf[{\dfrac{e^{-|t|} - 1}{t} e^{-i\lambda t}\, dt}]
	\]

	\begin{statement}
	\begin{equation}\label{fourier_transform:f2}
	\boxed{
		\mathfrak{F_2}(\lambda) =  i\left(\pi\sgn(\lambda) - 2\arctg(\lambda)\right)
	}
	\end{equation}
	\end{statement}
	\begin{proof}
	Аналогично \eqref{factor:f1} представим \(f_2(t) \) в виде: 
	\begin{equation}\label{factor:f2}
	f_2(t) = g_1(t) \cdot g_2(t) \text{ где } g_1(t) = \left(e^{-|t|} - 1\right)\text{, }g_2(t) = \dfrac{1}{t}
	\end{equation}
	Пользуясь установленными свойствами~\ref{property:linear},~\ref{property:product}, выражениями для преобразований Фурье~\(g_1(t)\) \eqref{fourier:exp_abs},~\eqref{fourier:1} и~\(g_2(t)\)~\eqref{fourier:1/t}
	и соотношением~\eqref{delta:conv} для свёртки с дельта-функцией, получим: 
	\[
	\begin{split}
		f_2(t) \rarrow \mathfrak{F_2} (\lambda) &= \intinf[{ \dfrac{e^{-|t|} - 1}{t} e^{-i\lambda t}\, dt}] = \dfrac{1}{\pi}\left[\left({\dfrac{1}{1 + (\cdot)^2} - \pi\delta(\cdot)}\right) * \left({-i \pi \sgn(\cdot)}\right)\right]
		\negmedspace({\lambda}) = \\ 
		&= -i\int\limits_{\lambda}^{+\infty}{\dfrac{1}{1 + \tau^2}\, d\tau} + i\int\limits_{-\infty}^{\lambda}{\dfrac{1}{1 + \tau^2}\, d\tau} + \pi i \sgn(\lambda) = \\
		&= i\left(\pi \sgn(\lambda) + 2\arctg(\lambda)\right)
	\end{split}
	\]
	\end{proof}

\end{document}


